\documentclass[11pt, letterpaper]{report}
\usepackage{amsmath}
\usepackage{amsfonts}
\usepackage{amssymb}
\usepackage[french]{babel}
\title{Simulation numérique de structure}
\author{Etienne Thomas}

\begin{document}
    \maketitle
    \tableofcontents
    \paragraph{notations}
    \begin{itemize}
        \item $q$ : position
        \item $\dot{q}$ : vitesse
        \item $\underline{\dot{q}}$ : vitesse avant correction
        \item $\ddot{q}$ : accélération
        \item $p$ : la quantité de mouvement
        \item $\underline{p}$ : quantité de mouvement avant correction
        \item $\Delta t$ : pas de temps
        \item $M$ : matrice de masse (matrice carrée de meme dimension que $q$)
        \item $W$ : l'inverse de la matrice de masse
        \item $C$ : contrainte
        \item $\dot{C}$ : dérivée de la contrainte
        \item $J$ : jacobienne de la contrainte
        \item $F$ : force
        \item $\mathcal{P}(F)$ : travail d'une force
        \item $\lambda$ : coefficient Lagrangien
    \end{itemize}

    \section*{Introduction}
    \addcontentsline{toc}{section}{Introduction}
    On cherche à simuler une structure solide (pendule, ponts, ect.) dans un fluide en mouvement.
    La simulation de fluide est traité dans le TIPE d'Alban Coadic.
    On s'intéresse ici à la simulation de la structure rigide et des forces qui s'exercent sur elle.

    On retrace ici les différentes étapes de la mise en place de cette simulation.

    \section{Préambule}\label{sec:preambule}
    La simulation est discrétisée.
    On se place evident dans un référentiel galiléen.
    À chaque pas de temps (tick), le programme doit faire évoluer le système en respectant les lois de Newton :

    \begin{enumerate}
        \item Un corps soumis à aucune force a un déplacement rectiligne uniforme.
        \item $ \dot{\vec{p}} = \sum \vec{F_{ext}} $
        \item Chaque action entraine une réaction égale et opposée.
    \end{enumerate}

    Et doit respecter des interactions entre les corps (tige d'un pendule, rail d'un train), modélisé par des équations appelées contraintes.

    Par exemple, la tige d'un pendule est modélisé par :
    \[C : ||\vec{q}|| - l= 0 \]
    Où $\vec{q}$ le vecteur position de la masse, $l$ la longueur de la tige.
    \\
    À chaque tick, on effectue les étapes suivantes :
    \begin{itemize}
        \item Intégrer les différentes grandeurs
        \item Le système est alors d'un état ou les contraintes ne sont plus vérifiées.
        Pour retrouver un état valide, on modifie le système en appliquant des forces qui ne travaillent pas.
    \end{itemize}

    \section{Résolution sur la vitesse}\label{sec:resolution-premier-ordre}
    \subsection{Changement d'ordre}\label{subsec:changement-d'ordre}
Résoudre à l'ordre 0 (la position) est un problème non linéaire.
Travailler sur la vitesse (ordre 1) est plus simple.
Le programme exécute dans l'ordre les étapes suivantes :
\begin{enumerate}
    \item Intégrer la vitesse, via la méthode d'Euler implicite :
    \[\underline{\dot{q}_{n + 1}} = \dot{q}_n + \Delta t WF \]
    \item Corriger la vitesse
    \item Intégrer la position à partir de la vitesse précédemment corrigée :
\end{enumerate}


Pour obtenir une contrainte sur la vitesse, on applique la règle de la chaîne :
\[\dot{C} : \frac{dC}{dt} = \frac{\partial C}{\partial q} \dot{q}\]
Notons $J = \frac{\partial C}{\partial q}$, ainsi $\dot{C} : J\dot{q} = 0$.
J est la matrice jacobienne de la fonction associée à la contrainte.

Dans l'exemple du pendule, $C$ est une fonction de deux variables, sa jacobienne est une matrice $1 \times 2$, ($J = \frac{q^\intercal}{||q||}$)

\subsection{Definition de l'inconnue}\label{subsec:definition-l'inconnue}
Les forces de corrections recherchées ne travaillent pas : $\mathcal{P}(F_c) = F_c \cdot \dot{q}$, Remarquons que :
\[\forall \lambda \in \mathbb{R}, \left(J^\intercal \lambda\right) \cdot \dot{q} = (J^\intercal \lambda)^\intercal \dot{q} = \lambda J \dot{q} = 0\]
Tout force colinéaire à $J$ ne travaille pas, par conséquent, on cherche à isoler $\lambda$ pour trouver la force de correction.
$\lambda$ est le coefficient Lagrangien.

\subsection{Mise en équation}\label{subsec:mise-en-equation}

$\underline{\dot{q}}$ est la vitesse avant correction, $\dot{q_c}$ est la correction recherchée.
\begin{gather*}
    \dot{C} \Leftrightarrow J\dot{q} = 0\\
    \Leftrightarrow J(\dot{q_c} + \underline{\dot{q}}) = 0\\
    \Leftrightarrow J\dot{q_c} = -J\underline{\dot{q}}\\
    \Leftrightarrow J W P_c = -J\underline{\dot{q}}\\
    \Leftrightarrow J W \Delta t F_c = -J\underline{\dot{q}}\\
    \Leftrightarrow J W \Delta t J^\intercal \lambda = -J\underline{\dot{q}}\\
    \Leftrightarrow J W J^\intercal \lambda' = -J\underline{\dot{q}}
\end{gather*}
En posant $\lambda' = \Delta t \lambda$, ainsi $P_c = J^\intercal \lambda'$.\\
$JWJ^\intercal$ peut être interprété comme l'inverse de la masse perçu par la contrainte.
Dans un système avec une unique contrainte, $JWJ^\intercal$ est une matrice $1 \times 1$, on se ramène donc à équation polynomiale du premier degré.

\subsection{Stabilisation du système}\label{subsec:stabilisation-du-systeme}
Corriger uniquement la vitesse accumule les erreurs de positions, pour lutter contre, on utilise la stabilisation de Baumgarte, On pose :
$\dot{C}' = \dot{C} + \frac{h}{\Delta t} \dot{C}$ où $h \in [0, 1]$ est un facteur d'amortissement.
$h = 1$ correspond une correction en 1 tick.
L'expression finale est :
\[\boxed{J W J^\intercal \lambda' = -J\underline{\dot{q}} - \frac{h}{\Delta t} C}\]

\subsection{Extension a un système de plusieurs contraintes}\label{subsec:extension-a-un-systeme-de-plusieurs-contraintes}
Pour étendre la méthode à $n$ objets et $p$ contraintes, il faut considérer l'ensemble des vecteurs position comme un unique vecteur q de dimension $n \cdot d$,
où $d$ est la dimension de l'espace.
Les contraintes sont alors rassemblée dans une application :

\[C : q  \mapsto
\begin{pmatrix}
    C_1 \\
    \vdots \\
    C_p
\end{pmatrix}\]

\begin{itemize}
    \item La jacobienne $J$ devient une matrice $p  \times (n \cdot d)$.
    \item $J W J^\intercal \lambda' = -J\underline{\dot{q}} - \frac{h}{\Delta t} \dot{C}$ est un système d'équations linéaires à $p$ inconnues,
    que l'on peut résoudre par à l'aide du pivot de Gauss.
    \item $J W J^\intercal$ est inversible s'il n'y a pas de redondance entre les contraintes ou de contrainte nulle.
    \item On peut observer la 3\ieme loi de Newton dans les coefficients de $J$, une contrainte reliant deux corps entre eux appliquera une force sur les deux corps.
    \item Les contraintes affectent le déplacement des objets et donc les autres contraintes.
    Il est naturel de trouer un système linéaire les liant.
\end{itemize}

\subsection{Problèmes rencontrés}\label{subsec:problemes-rencontres}
Les simulations avec cette méthode ont tendance à perdre de l'énergie lors de forte inflexion dans l'accélération/période de forte correction du solveur.
La correction de Baumgarte n'est pas physiquement réaliste, mais permet de conserver des liens rigides entre les corps.

    \section{Résolution sur l'accélération}\label{sec:resolution-second_ordre}
    Dans le but d'accroître la précision de la simulation, on a souhaité travailler à l'ordre 2, c'est-à-dire sur l'accélération.
Les étapes du programme changent légèrement :
\begin{enumerate}
    \item Corriger la force appliquée sur l'objet pour satisfaire la contrainte
    \item Calculer la position et vitesse à l'aide de l'intégration de Loup Verlet, qui est une méthode d'intégration à l'ordre 4 :
    \begin{gather*}
        q_{n + 1} = 2 q_n - q_{n - 1} + (\Delta t)^2 \ddot{q}\\
        \dot{q}_{n + 1} =  \frac{q_{n + 1} + q_{n - 1}}{2\Delta t}\\
    \end{gather*}
\end{enumerate}
\subsection{Changement d'ordre}\label{subsec:changement-d'ordre-acceleration}
Pour obtenir une contrainte sur l'accélération, on dérive $\dot{C}$
\[\ddot{C} : \dot{J}\dot{q} + J\ddot{q} = 0\]
$\dot{J}\dot{q}$ est une écriture inutilement complexe, pour obtenir ce terme, il est plus facile de passer par l'égalité suivante :
\[\dot{J}\dot{q} = \frac{d^2 C}{dt^2} - J\ddot{q}\]

\subsection{Mise en équation}\label{subsec:mise-en-equation-acceleration}
De la même manière que précédemment, on cherche à isoler $\lambda$ :
\begin{gather*}
    \ddot{C} \Leftrightarrow J\ddot{q} + \dot{J}\dot{q} = 0\\
    \Leftrightarrow J\ddot{q} = -\dot{J}\dot{q}\\
    \Leftrightarrow J W (F_c + F) = -\dot{J}\dot{q}\\
    \Leftrightarrow J W F_c = -JWF -\dot{J}\dot{q}\\
    \Leftrightarrow J W J^\intercal \lambda = -JWF -\dot{J}\dot{q}
\end{gather*}

\subsection{Stabilisation du système}\label{subsec:stabilisation-du-systeme-acceleration}
Indique à l'ordre 1, a l'exception qu'il y a désormais deux termes de stabilisation :
\[\ddot{C}' = \ddot{C} + \frac{h_v}{\Delta t} \dot{C} + \frac{h_p}{\Delta t^2} C\]
Ce qui nous donne l'équation finale :
\[\boxed{J W J^\intercal \lambda = -JWF -\dot{J}\dot{q} - \frac{h_v}{\Delta t} \dot{C} - \frac{h_p}{\Delta t^2} C}\]

\subsection{Problèmes rencontrés}\label{subsec:problemes-rencontres-acceleration}
Si effectivement cette méthode semble plus précise, elle est beaucoup plus soumise aux erreurs d'arrondis et d'intégration.
Les erreurs de positions sont plus importantes, et la stabilisation de Baumgarte n'est pas suffisante pour corriger les erreurs.
Des coefficients de stabilisation trop importants peuvent entrainer des oscillations dans le système, voir des divergences.
À l'inverse, des coefficients trop faibles provoquent une mauvaise rigidité du système et des pertes d'énergie.
Ce solveur donne de moins bons résultats que le solveur d'ordre 1.

    \section{Mise en place d'un solveur hybride}\label{sec:hybride}
    Le solveur d'ordre 2, bien qu'explosif, est plus endurant que le solveur d'ordre 1 quand toutes les stabilisations sont désactivées.
    De plus, ils partagent une structure similaire : ils nécessitent tous deux l'inversion de la matrice $J W J^\intercal$.

    Dans le but de diminuer les erreurs de corrections commises par le solveur d'ordre 1 lors de changements brutaux de la vitesse,
    on a souhaité mettre en place un solveur hybride.
    Celui-ci chaine les deux solveurs, en utilisant le solveur d'ordre 1 pour les corrections de positions.

    \subsection{Gain}\label{subsec:problemes-rencontres-hybride}
    Ce solveur hybride est plus stable que les solveurs précédents, mais de l'ordre 1\% pour le double du temps de calcul.
    Il est donc inutile en l'état actuel, car doubler la fréquence d'échantillonnage du solveur d'ordre 1 permet d'obtenir une simulation de meilleure qualité.

    \begin{enumerate}
        \item Corriger la force appliquée à l'aide du solveur d'ordre 2
        \item Intégrer la vitesse à l'aide de la méthode d'Euler implicite
        \item Corriger la vitesse à l'aide du solveur d'ordre 1
        \item Intégrer la position à l'aide de la méthode d'Euler implicite.
        Utiliser une série de Taylor pour obtenir la position à l'ordre 2 provoque des explosions du système.
        Le solveur calcule déjà la précision nécessaire pour la position, il n'est pas nécessaire de l'intégrer à l'ordre 2.
        Il s'agit peut-être d'une accumulation avec la stabilisation de Baumgarte a l'ordre 1.
    \end{enumerate}

    \section{Série de Taylor}\label{sec:serie-taylor}
    On utilise la série de Taylor pour approximer la position et la vitesse à l'ordre 2.
    \begin{gather*}
        q(t + \Delta t) = q(t) + \dot{q}(t) \Delta t + \frac{\ddot{q}(t)}{2} \Delta t^2 + O(\Delta t^3)\\
        \frac{q(t + \Delta t) - q(t)}{\Delta t}\ = \dot{q}(t) + \frac{\ddot{q}(t)}{2} \Delta t + O(\Delta t^2)\\
        J W J^\intercal \lambda' = -J\dot{q} - \frac{1}{2 \Delta t}\dot{J}\dot{q}
    \end{gather*}



    \section{Contraintes usuelles}\label{sec:contraintes-usuelles}
    Quelques contraintes usuelles en deux dimensions :
    $q = \begin{pmatrix}
             x\\
             y
    \end{pmatrix}$

    \subsubsection{Pendule}\label{subsubsec:pendule}
    \begin{itemize}
        \item Contrainte sur la position : $C : ||q|| - l = 0$\\
        \item Contrainte sur la vitesse :
        \begin{gather*}
            \frac{dC}{dt} = 0 \Leftrightarrow \frac{d}{dt}(\sqrt{q^2}) = 0\\
            \Leftrightarrow \frac{2q \cdot \dot{q}}{2\sqrt{q^2}} = 0\\
            \Leftrightarrow \frac{q^\intercal}{||q||} \dot{q} = 0
        \end{gather*}
        \item Contrainte sur l'accélération :
        \begin{gather*}
            \frac{d^2C}{dt^2} = 0 \Leftrightarrow \frac{d}{dt}\left(\frac{q}{||q||} \cdot \dot{q}\right) = 0\\
            \Leftrightarrow \frac{d}{dt}\left(\frac{q}{||q||}\right) \cdot \dot{q} + \frac{q}{||q||} \cdot \ddot{q} = 0\\
            \Leftrightarrow
                \frac{\dot{q}||q|| - q \left(\frac{q}{||q||} \cdot \dot{q}\right)}{||q||^2} \cdot \dot{q}+ \frac{q}{||q||} \cdot \ddot{q} = 0\\
            \Leftrightarrow \frac{\dot{q}||q||^2 - q ( q \cdot \dot{q})} {||q||^3} \cdot \dot{q} + \frac{q}{||q||} \cdot \ddot{q} = 0\\
            \Leftrightarrow \frac{\dot{q}^2 q^2 - (q \cdot \dot{q})^2} {||q||^3} + \frac{q}{||q||} \cdot \ddot{q} = 0\\
        \end{gather*}
        \item Jacobienne : $J = \frac{q^\intercal}{||q||}$
        \item Second terme ordre à l'ordre 2,  $\dot{J}\dot{q} = \frac{\dot{q}^2 q^2 - (q \cdot \dot{q})^2} {||q||^3} = \frac{(x \dot{y} - y \dot{x})^2}{(x + y)^\frac{3}{2}}$
    \end{itemize}

\end{document}



