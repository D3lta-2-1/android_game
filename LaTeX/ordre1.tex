\subsection{Changement d'ordre}\label{subsec:changement-d'ordre}
Résoudre à l'ordre 0 (la position) est un problème non linéaire.
Travailler sur la vitesse (ordre 1) est plus simple.
Le programme exécute dans l'ordre les étapes suivantes :
\begin{enumerate}
    \item Intégrer la vitesse, via la méthode d'Euler implicite :
    \[\underline{\dot{q}_{n + 1}} = \dot{q}_n + \Delta t WF \]
    \item Corriger la vitesse
    \item Intégrer la position à partir de la vitesse précédemment corrigée :
\end{enumerate}


Pour obtenir une contrainte sur la vitesse, on applique la règle de la chaîne :
\[\dot{C} : \frac{dC}{dt} = \frac{\partial C}{\partial q} \dot{q}\]
Notons $J = \frac{\partial C}{\partial q}$, ainsi $\dot{C} : J\dot{q} = 0$.
J est la matrice jacobienne de la fonction associée à la contrainte.

Dans l'exemple du pendule, $C$ est une fonction de deux variables, sa jacobienne est une matrice $1 \times 2$, ($J = \frac{q^\intercal}{||q||}$)

\subsection{Definition de l'inconnue}\label{subsec:definition-l'inconnue}
Les forces de corrections recherchées ne travaillent pas : $\mathcal{P}(F_c) = F_c \cdot \dot{q}$, Remarquons que :
\[\forall \lambda \in \mathbb{R}, \left(J^\intercal \lambda\right) \cdot \dot{q} = (J^\intercal \lambda)^\intercal \dot{q} = \lambda J \dot{q} = 0\]
Tout force colinéaire à $J$ ne travaille pas, par conséquent, on cherche à isoler $\lambda$ pour trouver la force de correction.
$\lambda$ est le coefficient Lagrangien.

\subsection{Mise en équation}\label{subsec:mise-en-equation}

$\underline{\dot{q}}$ est la vitesse avant correction, $\dot{q_c}$ est la correction recherchée.
\begin{gather*}
    \dot{C} \Leftrightarrow J\dot{q} = 0\\
    \Leftrightarrow J(\dot{q_c} + \underline{\dot{q}}) = 0\\
    \Leftrightarrow J\dot{q_c} = -J\underline{\dot{q}}\\
    \Leftrightarrow J W P_c = -J\underline{\dot{q}}\\
    \Leftrightarrow J W \Delta t F_c = -J\underline{\dot{q}}\\
    \Leftrightarrow J W \Delta t J^\intercal \lambda = -J\underline{\dot{q}}\\
    \Leftrightarrow J W J^\intercal \lambda' = -J\underline{\dot{q}}
\end{gather*}
En posant $\lambda' = \Delta t \lambda$, ainsi $P_c = J^\intercal \lambda'$.\\
$JWJ^\intercal$ peut être interprété comme l'inverse de la masse perçu par la contrainte.
Dans un système avec une unique contrainte, $JWJ^\intercal$ est une matrice $1 \times 1$, on se ramène donc à équation polynomiale du premier degré.

\subsection{Stabilisation du système}\label{subsec:stabilisation-du-systeme}
Corriger uniquement la vitesse accumule les erreurs de positions, pour lutter contre, on utilise la stabilisation de Baumgarte, On pose :
$\dot{C}' = \dot{C} + \frac{h}{\Delta t} \dot{C}$ où $h \in [0, 1]$ est un facteur d'amortissement.
$h = 1$ correspond une correction en 1 tick.
L'expression finale est :
\[\boxed{J W J^\intercal \lambda' = -J\underline{\dot{q}} - \frac{h}{\Delta t} C}\]

\subsection{Extension a un système de plusieurs contraintes}\label{subsec:extension-a-un-systeme-de-plusieurs-contraintes}
Pour étendre la méthode à $n$ objets et $p$ contraintes, il faut considérer l'ensemble des vecteurs position comme un unique vecteur q de dimension $n \cdot d$,
où $d$ est la dimension de l'espace.
Les contraintes sont alors rassemblée dans une application :

\[C : q  \mapsto
\begin{pmatrix}
    C_1 \\
    \vdots \\
    C_p
\end{pmatrix}\]

\begin{itemize}
    \item La jacobienne $J$ devient une matrice $p  \times (n \cdot d)$.
    \item $J W J^\intercal \lambda' = -J\underline{\dot{q}} - \frac{h}{\Delta t} \dot{C}$ est un système d'équations linéaires à $p$ inconnues,
    que l'on peut résoudre par à l'aide du pivot de Gauss.
    \item $J W J^\intercal$ est inversible s'il n'y a pas de redondance entre les contraintes ou de contrainte nulle.
    \item On peut observer la 3\ieme loi de Newton dans les coefficients de $J$, une contrainte reliant deux corps entre eux appliquera une force sur les deux corps.
    \item Les contraintes affectent le déplacement des objets et donc les autres contraintes.
    Il est naturel de trouer un système linéaire les liant.
\end{itemize}

\subsection{Problèmes rencontrés}\label{subsec:problemes-rencontres}
Les simulations avec cette méthode ont tendance à perdre de l'énergie lors de forte inflexion dans l'accélération/période de forte correction du solveur.
La correction de Baumgarte n'est pas physiquement réaliste, mais permet de conserver des liens rigides entre les corps.