Dans le but d'accroître la précision de la simulation, on a souhaité travailler à l'ordre 2, c'est-à-dire sur l'accélération.
Les étapes du programme changent légèrement :
\begin{enumerate}
    \item Corriger la force appliquée sur l'objet pour satisfaire la contrainte
    \item Calculer la position et vitesse à l'aide de l'intégration de Loup Verlet, qui est une méthode d'intégration à l'ordre 4 :
    \begin{gather*}
        q_{n + 1} = 2 q_n - q_{n - 1} + (\Delta t)^2 \ddot{q}\\
        \dot{q}_{n + 1} =  \frac{q_{n + 1} + q_{n - 1}}{2\Delta t}\\
    \end{gather*}
\end{enumerate}
\subsection{Changement d'ordre}\label{subsec:changement-d'ordre-acceleration}
Pour obtenir une contrainte sur l'accélération, on dérive $\dot{C}$
\[\ddot{C} : \dot{J}\dot{q} + J\ddot{q} = 0\]
$\dot{J}\dot{q}$ est une écriture inutilement complexe, pour obtenir ce terme, il est plus facile de passer par l'égalité suivante :
\[\dot{J}\dot{q} = \frac{d^2 C}{dt^2} - J\ddot{q}\]

\subsection{Mise en équation}\label{subsec:mise-en-equation-acceleration}
De la même manière que précédemment, on cherche à isoler $\lambda$ :
\begin{gather*}
    \ddot{C} \Leftrightarrow J\ddot{q} + \dot{J}\dot{q} = 0\\
    \Leftrightarrow J\ddot{q} = -\dot{J}\dot{q}\\
    \Leftrightarrow J W (F_c + F) = -\dot{J}\dot{q}\\
    \Leftrightarrow J W F_c = -JWF -\dot{J}\dot{q}\\
    \Leftrightarrow J W J^\intercal \lambda = -JWF -\dot{J}\dot{q}
\end{gather*}

\subsection{Stabilisation du système}\label{subsec:stabilisation-du-systeme-acceleration}
Indique à l'ordre 1, a l'exception qu'il y a désormais deux termes de stabilisation :
\[\ddot{C}' = \ddot{C} + \frac{h_v}{\Delta t} \dot{C} + \frac{h_p}{\Delta t^2} C\]
Ce qui nous donne l'équation finale :
\[\boxed{J W J^\intercal \lambda = -JWF -\dot{J}\dot{q} - \frac{h_v}{\Delta t} \dot{C} - \frac{h_p}{\Delta t^2} C}\]

\subsection{Problèmes rencontrés}\label{subsec:problemes-rencontres-acceleration}
Si effectivement cette méthode semble plus précise, elle est beaucoup plus soumise aux erreurs d'arrondis et d'intégration.
Les erreurs de positions sont plus importantes, et la stabilisation de Baumgarte n'est pas suffisante pour corriger les erreurs.
Des coefficients de stabilisation trop importants peuvent entrainer des oscillations dans le système, voir des divergences.
À l'inverse, des coefficients trop faibles provoquent une mauvaise rigidité du système et des pertes d'énergie.
Ce solveur donne de moins bons résultats que le solveur d'ordre 1.